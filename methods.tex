\dfosection{Materials and methods}{}

% Cover page
\dfosubsection{Cover page}{}

The cover page must be produced using the DFO template. Choose the appropriate file for your language and report series under \texttt{CoverMaterial/doc/} and open it using Microsort Word. Edit the required fields and export the document as a \texttt{pdf} file. Save it as \texttt{CoverPage.pdf} under \texttt{CoverMaterial/}. Compile \texttt{main.tex} and your cover page should be updated.

% Front matter
\dfosubsection{Front matter}{}

After the cover page comes the series description. This must also be produced using the DFO templates. Since they require no editing, \texttt{pdf} versions have been created for each series type and are included with this report template. To select the correct series description, open \texttt{settings/reporttype.tex} and un-comment the appropriate code block. Recompile \texttt{main.tex} and your series description should be updated.

Information presented on the title and colophon pages as well as the correct citations then needs to be edited. These settings are found in the file \texttt{settings/info.tex}. Once they have been edited, Recompile \texttt{main.tex} and your title page, colophon page and citations under the abstract and résumé should display the correct information.


% Contents
\dfosubsection{Content listings}{}

Depending on the report, a table of contents, list of figures and list of tables may or may not be necessary. All are automatically generated by default. To exclude those not needed, open \texttt{main.tex} and scroll to the content listings block. There, comment lines inserting the tables or lists to exclude.

% Body
\dfosubsection{Document body}{}

This template already includes \texttt{.tex} files for each section of the document structure required by the guide to authors \citep{GUIDE}. The body of the text should be written to the following files.

\begin{itemize}
    \item abstract.tex
    \item introduction.tex
    \item methods.tex
    \item results.tex
    \item results\_subsection\_*.tex
    \item discussion.tex
    \item acknowledgements.tex
\end{itemize}

Sections, subsections and sub-subsections can be numbered or unnumbered. The headings must however be entered using the \texttt{dfosection\{heading\}\{file.tex (optional)\}} and \texttt{dfostarsection\{heading\}\{file.tex (optional)\}} macros to maintain the table of contents order. Similar macros exist for sub and sub-subsections with the "sub" prefix added to the word "section" in the macro names. Unnumbered sections use the "star" macros and numbered sections use the others. These macros set the label of to \texttt{sec:heading}, \texttt{ssec:heading}, or \texttt{sssec:heading} for sections, subsections and sub-subsections respectively.

% Internal references
\dfosubsection{Internal references}{}

References to figures, tables and numbered sections can be made as usual using the \texttt{ref} command (e.g., Figure~\ref{fig:ex}, Table~\ref{tab:example}, Subsection~\ref{ssec:Internal references}). Sections which are not numbered can be reference by name using the \texttt{nameref} command (e.g., the~\nameref{sec:Acknowledgements} section).

% External references
\dfosubsection{External references}{}

Reference are expected to be managed using \texttt{bibtex}. Overwrite the file \texttt{library.bib} with your own bibliographic reference file, recompile and your library should be available to the citation commands. The \texttt{bib} file can be produced manually but can also be (and often is) generated using reference management software like Mendeley or Zotero.

The LaTeX package used for reference management is \texttt{apacite} with some customisation such as to match reference list styles requested by the \href{https://cdnsciencepub.com/journal/cjfas}{Canadian Journal of Fisheries and Aquatic Science}'s author guidelines. References should be entered using the \texttt{cite} or \texttt{citep} commands. Examples of references to several types of common documents are shown in Table~\ref{tab:ref}.

\begin{table}[ht]
    \centering
    \caption{Examples of references to several types of document and how they will appear in text using automated citation commands.}
    \label{tab:ref}
    \begin{tabular}{lll}
    \hline\hline
        Case & Using \texttt{cite} & Using \texttt{citep} \\ \hline
        1 author article & \cite{Doe2020} & \citep{Doe2020} \\
        2 authors article & \cite{Doe2019} & \citep{Doe2019} \\
        3 or more authors article & \cite{Doe2018} & \citep{Doe2018} \\
        Book (authors) & \cite{Doe2017} & \citep{Doe2017} \\
        Book (editors) & \cite{Doe2016} & \citep{Doe2016} \\
        Book section & \cite{Doe2015} & \citep{Doe2015} \\
        Book in series & \cite{Doe2014} & \citep{Doe2014} \\
        Ph. D. thesis & \cite{Doe2000} & \citep{Doe2000}\\
        M. Sc. thesis & \cite{Brown2005} & \citep{Brown2005}\\
        DFO science report & & \\ \hline
    \end{tabular}

\end{table}



% Editing formatting options
\dfosubsection{Formatting options}{}

Although instructions are purposefully omitted for changing the document's format, most formatting options can be found inside the file \texttt{settings/formatting.tex}, with the exception of the base font size specified at the top of \texttt{main.tex}.

\clearpage