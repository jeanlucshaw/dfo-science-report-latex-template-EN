% ==========================================================
%
% This block should not be edited
%
% ==========================================================
\clearpage
\dfosection{Introduction}{} % do no use the second parameter of \dfosection for the introduction
%\section{Introduction}
\pagenumbering{arabic}      % otherwise arabic numerals may start on the wrong page
% ==========================================================
%
%
%
% ==========================================================

% Text of the introduction here
LaTeX is an efficient tool for the production of reports and scientific communication. Many of its features streamline the writing process. For example, links to figure images outside the document are used for illustrations such that if a figure is changed, it will be updated automatically in the final document. This avoids image file duplication and mistakes due to multiple existing versions. LaTeX also offers robust tools for internal and external referencing, removing the possibility of numbering mistakes in references to figures, tables, sections as well as mistakes in the numbering of lists of figures, lists of tables and tables of content. Finally, LaTeX limits the user's possible formatting choices, making it perfect for delivering documents of uniform formatting such as is required from the DFO science report series.

Once it is mastered, LaTeX saves time for authors by removing most tasks related to editing and it saves time for editors by eliminating the need for checking many tedious details which have been safely automated. Though its use is not currently widespread throughout the scientific community, it should be an option available to DFO scientist. These were the reasons motivating the present science report template.

The methods section of this document serves as a step by step guide to set up this template for your project, and the results section contains examples of some elements likely to be used by most users. Files are also included for inserting a discussion and acknowledgements sections but are left empty. Note that the user's experience with LaTeX is assumed. For installation and tutorials refer to \href{http://tug.org/}{this page}. Comments on improvements and bugs are welcome. Contact the author at Jean-Luc.Shaw@dfo-mpo.gc.ca.
